### **Resume Model IFPOM dan Metodologi Validasi**

**Integrated Flexible Portfolio Optimization Model (IFPOM)** telah dikembangkan sebagai model optimasi portofolio proyek yang mempertimbangkan **interdependensi proyek, strategi pendanaan hybrid, dan pengambilan keputusan berbasis Multi-State Dynamic Decision Making (MSDDM)**. Hingga tahap ini, model telah dirancang untuk lebih fleksibel dan adaptif dibandingkan model konvensional.

---

## **1. Ketidakpastian dalam Biaya dan Pendanaan**

Dalam pengelolaan portofolio proyek teknologi informasi dan kecerdasan buatan (TI/AI), terdapat beberapa sumber ketidakpastian dalam biaya dan pendanaan yang harus diperhatikan secara khusus. Ketidakpastian tersebut meliputi fluktuasi biaya pengembangan yang dipengaruhi oleh kompleksitas teknis dan teknologi yang digunakan, variabilitas dalam biaya infrastruktur seperti server cloud dan kebutuhan komputasi, ketidakpastian terkait sumber pendanaan proyek seperti ketersediaan dana dari investor atau pendanaan internal, serta ketidakpastian dalam tingkat pengembalian investasi (Return on Investment - ROI) akibat fluktuasi pasar dan tingkat adopsi teknologi.

---

Referensi: Epha Diana Supandi (2020). Pengembangan Model Portofolio Mean-Variance Melalui Metode Estimasi Robust dan Optimasi Robust. Universitas Gadjah Mada.
